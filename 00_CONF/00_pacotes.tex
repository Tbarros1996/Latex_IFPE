% Instituto Federal de Pernambuco - Campus Recife
% Departamento de Engenharia Mecânica
% LaTeX - 3.14
% Desenvolvido por Thiago Barros
% Modelo de Relatório em LaTex Para Trabalhos Acadêmicos - e Trabalho de Conclusão de Curso

% Lista de Pacotes


\usepackage[utf8]{inputenc}  % Permite a entrada de caracteres UTF-8 (acentos, etc.)
\usepackage[left=3cm,top=3cm,right=2cm,bottom=2cm]{geometry}  % Define as margens do documento
\usepackage[brazil]{babel}  % Localização e hifenização para o português do Brasil
\usepackage{graphicx}  % Permite a inclusão de figuras
\usepackage{amsmath}  % Pacote matemático da AMS (American Mathematical Society)
\usepackage{amsfonts}  % Fontes matemáticas da AMS
\usepackage{amssymb}  % Símbolos matemáticos da AMS
\usepackage{mathtools}  % Extensão do pacote amsmath com melhorias
\usepackage{indentfirst}  % Adiciona recuo no primeiro parágrafo de cada seção
\usepackage{ragged2e}  % Fornece o comando \justify para justificar texto
\usepackage{caption}  % Personaliza a formatação de legendas de figuras e tabelas
\usepackage{pdfpages}  % Permite a inclusão de páginas PDF externas

\usepackage{fancyhdr}  % Personaliza cabeçalhos e rodapés do documento
\pagestyle{fancy}
\fancyhf{}
\renewcommand{\headrulewidth}{0pt}
\rhead[R]{\thepage}


% Fonte Arial
%\usepackage{uarial} % Use a fonte Arial
%\renewcommand{\familydefault}{\sfdefault} 


% Fonte Times New Roman
\usepackage{times}  % Usa a fonte Times New Roman



\usepackage{lipsum} % Para gerar texto de exemplo
\usepackage{enumerate}  % Personaliza a numeração de listas enumeradas
\usepackage[skins,theorems]{tcolorbox}  % Permite a criação de caixas de texto personalizadas
\usepackage{cancel}  % Permite cancelar (barrar) termos em equações
\usepackage{listings}  % Permite a inclusão de códigos fonte de várias linguagens
\usepackage{color}  % Fornece cores para usar em documentos

\usepackage{microtype}  % Melhora o ajuste e o espaçamento de caracteres
\usepackage{tabularx}  % Extensão do ambiente tabular com larguras automáticas
\usepackage{titlesec}  % Personaliza títulos de seções, capítulos, etc.
%\usepackage{tocbibind}  % Adiciona a tabela de conteúdos, lista de figuras, etc., ao índice
\usepackage{siunitx}  % Permite a formatação fácil de unidades SI
\usepackage{physics}  % Fornece comandos úteis para notação física e matemática
\usepackage{tikz}  % Pacote para criar gráficos vetoriais
\usepackage[outline]{contour}  % Adiciona contornos a texto
\usepackage{setspace}  % Controla o espaçamento entre linhas